\addcontentsline{toc}{chapter}{Acknowledgements}

\chapter*{Acknowledgements}

%not only the giants whose shoulders provided the proverbial support for this work, by also people who were beside me throughout this endeavour, who supported and encouraged me

First and foremost, I would like to acknowledge Prof. Ruslan Mitkov, the heed of the Research Group in Computational Linguistics at the University of Wolverhampton, who is very good at creating opportunities for other people. For me, this PhD programme was a wonderful chance to convert a hobby into a full-time occupation and to considerably progress and achieve personal goals that I had little hope to achieve otherwise. It applies to many other academic activities apart from the PhD research: supporting the Master's programme, working for the Journal of Natural Language Engineering, organising conferences. 

I want to express my belated appreciation of the diligence and consistency of my first-year supervisory team: Prof. Constantin Orasan and Dr. Sara Moze. They were key to installing me on the right track and mood for this research. 

Special thanks go to Andrey Kutuzov for all the inspiration, and mocking, and practical help; for countless discussions and genuine curiosity. Above all, I see him as a person who initiated my interest in empirical linguistics in the first place.

I am grateful to Serge Sharoff for the human warmth and kindness, for starting conversations, for his keen interest in my developments and for the shared linguistic passions. He is one of the those people who would not cringe at the perspective of talking about research.

Big thanks to Ekaterina Lapshiniva-Koltunski, my co-author in many publications, for the shared interest in the translationese studies and not begrudging the time to discuss it. suggesting useful and relevant reading, for   

I should mention Vasilisa Grosheva, my eldest daughter for tolerance when being shamelessly used as a sounding board to fly my research ideas. It meant a lot to me and in the long run helped to shape this thesis. And also, for silently taking over the household chores.
% I am guilting of having her read parts of this thesis as another pair of eyes, along with taking care of younger siblings and other animals.  
% kids for always lending an ear for my lengthy talks; overused as a sounding board.

I am grateful to Evgeniy Medvedev, my brother, who had an unwavering faith in me and offered useful support (including computational) and encouragement throughout the years of my studentship. This support meant a lot to me. 

Although the COVID pandemic interfered midway, I was happy to share the Wolverhampton office and many lunches with my colleagues: Alistair Plum, Shiva Taslimipoor, Omid Rohanian, Rocío Caro, Tharindu Ranasinghe. These people moulded my attitudes and experiences in this research field and supplied their bits that in the end joined as a jigsaw (or not). There is a special person in the RGCL office who always made me feel welcome and valued: April Harper, the most considerate and empathetic person I met. I think she is that key player, a constant in the RGCL team who glues together the pieces and smoothens the processes.

Finally, I want to embrace all the people in the NLP and the computational linguistics communities who do not see reviewing as a useless and formal chore, but believe that they are contributing to the progress in the field. Everybody knows that there is nothing more rewarding for an author than a knowledgeable, understanding and well-wishing critic who can suggest better options and ask the right questions. 

%We would like to acknowledge Donald Craig at Memorial University,
%Newfoundland who published the meta-thesis on which this template is
%based. You can find Donald's work on his web site, here:
%\url{http://www.cs.mun.ca/~donald/metathesis/}.

