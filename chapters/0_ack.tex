\addcontentsline{toc}{chapter}{Acknowledgements}

\chapter*{Acknowledgements}

%not only the giants whose shoulders provided the proverbial support for this work, by also people who were beside me throughout this endeavour, who supported and encouraged me

First and foremost, I would like to acknowledge Prof. Ruslan Mitkov, the head of the Research Group in Computational Linguistics at the University of Wolverhampton, who is known for raising a number of outstanding researchers, and who has a strong talent for creating opportunities. For me, this PhD programme was a lucky chance to convert a hobby into a full-time occupation and to achieve personal goals that I had little hope to accomplish otherwise. I am grateful for the very real support and advice Prof. Mitkov extended to me throughout my stay in Wolverhampton. %It applies to many other academic activities apart from the PhD research: supporting the Master's programme, working for the Journal of Natural Language Engineering, organising conferences. 
I was also lucky to have Prof. Gloria Corpas Pastor, the head of the \textit{Lexytrad} research group, University of Malaga, as a supervisor. She was the major source of inspiration and reassurance for me. Her energy, perseverance and work ethics combined with a friendly and approachable manner are impressive. 
Most heartfelt gratitude goes to Dr Le An Ha for the many challenging discussions and disagreements, for being very resourceful in helping me at the writing-up stage. He helped me wrap my mind around complex technical concepts by asking pointed questions and demanding detailed explanations. 

I want to express my belated appreciation of the diligence and consistency of my first-year supervisory team: Prof. Constantin Orasan and Dr Sara Moze. They were key to installing me on the right track and mood for this research. 

Special thanks go to Andrey Kutuzov for all the inspiration, and mocking, and many overnight vigils solving yet another technical task; for countless discussions and genuine curiosity. Above all, I see him as a person who initiated and fostered my interest in empirical linguistics many years ago, and who helped me overcome the initial intimidatingly steep learning curve.

I am grateful to Serge Sharoff for the human warmth and kindness, for starting conversations, for his keen interest in my developments and for the shared linguistic passions. He is one of those people who would not cringe at the perspective of talking about research over the weekend.

Big thanks to Ekaterina Lapshinova-Koltunski, my co-author in many publications, for the shared interest in translationese studies and willingness to discuss it. She is amazingly quick with providing links to relevant reading in the nick of the time, and I have learnt quite a few tricks of the academic trade from her. 

Although the COVID pandemic interfered midway, I was happy to share the Wolverhampton office and many lunches with my colleagues: Alistair Plum, Rocío Caro, Tharindu Ranasinghe, Omid Rohanian, Shiva Taslimipoor, Emma Franklin, Kanishka Silva, Fred Blain, Damith Mullage and others. These people moulded my attitudes and experiences in this research field, and each contributed their practical tips. More importantly, they were there for me when I needed to talk things over, both academically and life-wise, supporting my psychological well-being. Thanks for the laughs, going-outs, trips, parties and office times together. You will always be an unalienable part of this PhD, whether you want it or not. 
%There is a special person in the RGCL office who always made me feel welcome and valued: April Harper, the most considerate and empathetic person I met. To me, she was that key player in the RGCL team who glued together the pieces and smoothened many processes.

I hope for an understanding on the part of my children, who had to become more independent a bit earlier than expected. 
%I am afraid that they were a bit overused as a sounding board to fly my research ideas. 
Thank you for helping me stay in touch with reality, for silently taking over the household duties and for taking care of me. I am grateful to Evgeniy Medvedev, my brother, who had an unwavering faith in me and offered support and encouragement throughout the years of my studentship. This meant a lot to me. 

Finally, I want to embrace all the people in the NLP and computational linguistics communities who do not see reviewing as a useless and formal chore, but believe that they contribute to the progress in the field. There is nothing more rewarding for an author than a knowledgeable, understanding and well-wishing critic who can suggest better options and ask the right questions. 
